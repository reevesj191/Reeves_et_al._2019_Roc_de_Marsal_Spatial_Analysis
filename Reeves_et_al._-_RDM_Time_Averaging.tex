% !TeX program = pdfLaTeX
\documentclass[smallextended]{svjour3}       % onecolumn (second format)
%\documentclass[twocolumn]{svjour3}          % twocolumn
%
\smartqed  % flush right qed marks, e.g. at end of proof
%
\usepackage{amsmath}
\usepackage{graphicx}
\usepackage[utf8]{inputenc}

\usepackage[hyphens]{url} % not crucial - just used below for the URL
\usepackage{hyperref}
\providecommand{\tightlist}{%
  \setlength{\itemsep}{0pt}\setlength{\parskip}{0pt}}

%
% \usepackage{mathptmx}      % use Times fonts if available on your TeX system
%
% insert here the call for the packages your document requires
%\usepackage{latexsym}
% etc.
%
% please place your own definitions here and don't use \def but
% \newcommand{}{}
%
% Insert the name of "your journal" with
% \journalname{myjournal}
%

%% load any required packages here



\usepackage{lineno}
\usepackage{longtable, booktabs}
\usepackage{setspace}\doublespacing
\journalname{Anthropological and Archaeological Science}

\begin{document}

\title{Measuring Spatial Structure in Time-Averaged Deposits }
 \subtitle{Insights from Roc de Marsal, France} 

    \titlerunning{Measuring Spatial Structure in Time-Averaged Deposits}

\author{  Jonathan S. Reeves \and  Shannon P. McPherron \and  Vera Aldeias \and  Harold L. Dibble \and  Paul Goldberg \and  Dennis Sandgathe \and  Alain Turq \and  }


\institute{
        Jonathan S. Reeves \at
     George Washington University, Department of Anthropology, 2110 G Street
 North West, Washington, District of Columbia, 20052, USA \\
     \email{\href{mailto:jsreeves@gwu.edu}{\nolinkurl{jsreeves@gwu.edu}}}  %  \\
%             \emph{Present address:} of F. Author  %  if needed
    \and
        Shannon P. McPherron \at
     Department of Human Evolution, Max Planck Institute for Evolutionary
 Anthropology, DeutscherPlatz 6, Leipzig, D-04177, GERMANY \\
     \email{\href{mailto:mcpherron@eva.mpg.de}{\nolinkurl{mcpherron@eva.mpg.de}}}  %  \\
%             \emph{Present address:} of F. Author  %  if needed
    \and
        Vera Aldeias \at
     Interdisciplinary Center for Archaeology and the Evolution of Human
 Behaviour, FCHS, Universidade do Algarve, Campus de Gambelas, 8005-139,
 PORTUGAL \\
     \email{\href{mailto:veraldeias@gmail.com}{\nolinkurl{veraldeias@gmail.com}}}  %  \\
%             \emph{Present address:} of F. Author  %  if needed
    \and
        Harold L. Dibble \at
     Department of Anthropology, University of Pennsylvania, Philadelphia,
 PA, USA \\
     \email{\href{mailto:hdibble@sas.upenn.edu}{\nolinkurl{hdibble@sas.upenn.edu}}}  %  \\
%             \emph{Present address:} of F. Author  %  if needed
    \and
        Paul Goldberg \at
     Centre for Archaeological Science (CAS), School of Earth and
 Environmental Sciences, University of Wollongong, Northfields Avenue,
 Wollongong, NSW 2522 Australia and Institute for Archaeological
 Sciences, University of Tübingen, Rümelinstr. 23, 72070 Tübingen
 Germany, Interdisciplinary Center for Archaeology and the Evolution of
 Human Behaviour, FCHS, Universidade do Algarve, Campus de Gambelas,
 8005-139, PORTUGAL \\
     \email{\href{mailto:goldberg@uow.edu.au}{\nolinkurl{goldberg@uow.edu.au}}}  %  \\
%             \emph{Present address:} of F. Author  %  if needed
    \and
        Dennis Sandgathe \at
     Department of Archaeology and Human Evolutionary Studies Program, Simon
 Fraser University, Vancouver, CANADA \\
     \email{\href{mailto:dms@sfu.ca}{\nolinkurl{dms@sfu.ca}}}  %  \\
%             \emph{Present address:} of F. Author  %  if needed
    \and
        Alain Turq \at
     None \\
     \email{\href{mailto:alain.turq@orange.fr}{\nolinkurl{alain.turq@orange.fr}}}  %  \\
%             \emph{Present address:} of F. Author  %  if needed
    \and
    }

\date{Received: date / Accepted: date}
% The correct dates will be entered by the editor


\maketitle

\begin{abstract}
The use of space, both at the landscape and the site level, is
considered an important aspect of hominin adaptations that changed
through time. At the site level, spatial analyses are typically
conducted on deposits thought to have a high degree of temporal
resolution. Sites with highly time-averaged deposits are viewed as
inferior for these analyses because repeated site visits obscure
individual behavioral events. To the contrary, here we take the view
that behaviors that repeat themselves in a spatially structured way
through time are exactly the kinds of behaviors that are potentially
significant at an evolutionary time scale. In this framework, time
averaging is seen not as a hindrance but rather as a necessary condition
for viewing meaningful behavior. To test whether such patterning is
visible in time-averaged deposits, we use spatial statistics to analyze
a number of indices designed to measure lithic production, use and
discard behaviors in a multi-layer, late Neandertal cave site in
southwest France. We find that indeed some such patterning does exist,
and thus sites with highly time-averaged deposits have the potential to
contribute to our understanding of how hominin use of space varied
through time. This is useful because a great many archaeological sites
have highly time-average deposits. Interpreting the spatial patterning
will likely require modeling to create expectations in time-averaged and
likely emergent contexts such as these.
\\
\keywords{
        time-averaging \and
        Paleolithic \and
        palimpsets \and
        Roc de Marsal \and
        Middle Paleolithic \and
    }


\end{abstract}


\def\spacingset#1{\renewcommand{\baselinestretch}%
{#1}\small\normalsize} \spacingset{1}


\pagebreak
\linenumbers
\doublespacing

\textbf{Introduction}

The spatial structure of stone artifacts, fauna, and other
archaeological features (e.g.~hearths) provide a unique window into how
past people conceptualized and organized their behaviors in space
(Aldeias et al. 2012; Alperson-Afil 2008; Alperson-Afil 2017; Bamforth,
Becker, and Hudson 2005; Clark 2016; Henry et al. 2004; Pettitt 1997;
Vaquero and Pastó 2001; Yellen 1977). In turn, these patterns ultimately
reflect the ways in which humans interacted with their physical
environment. Intra-site spatial analyses thus provide the opportunity to
gain important insights into the behavior of Paleolithic peoples (Kroll
and Price 1991; Yvorra 2003; Henry et al. 2004; Henry 2012; Oron and
Goren-Inbar 2014; Mallol and Hernández 2016; Gopher et al. 2016).
Closely tied with the advent of processual archaeology in the 1960s,
spatial analyses utilized the material traces of modern forager activity
to interpret distributions of artifacts from archaeological sites in
behavioral terms (e.g Yellen 1977; Binford 1978; Gould and Yellen 1987;
Simms and Heath 1990; O'Connell, Hawkes, and Jones 1991; Audouze and
Enloe 1997). This approach focuses on identifying discrete ``zones'' of
activities whose arrangement in space provide insight into the spatial
organization of past people's behavior (Clark 2017). Since the 1960s,
spatial analysis has seen rapid methodological improvements in both data
collection and analysis. Plumb bobs and meter-sticks have given way to
devices capable of quickly and accurately mapping whole archaeological
assemblages in three-dimensional space (Reed et al. 2015; Wheatley and
Gillings 2013; McPherron, Dibble, and Goldberg 2005), and the increased
usability and accessibility of geographic information systems in the
last years provide a multitude of new tools by which spatial
archaeological data can be visualized and analyzed (Abe et al. 2010;
Benito-Calvo and Torre 2011; Wheatley and Gillings 2013; Machado et al.
2016). Simple distribution maps have been replaced with a variety of
point pattern analyses, geospatial statistics, and multivariate analysis
(Alperson-Afil et al. 2009; Merrill and Read 2010; Yvorra 2003).

Spatial analysis in Paleolithic studies has, however, slowed in recent
decades (Clark 2016; Clark 2017). Despite the fact that the
sophistication and the number of both documentation and analytical tools
have never been greater, archaeologists still grapple with some
fundamental issues regarding the formation of the archaeological record.
Although ethnographic data has been a primary analog for interpreting
the archaeological record, these data typically represent a time-scale
of hours to months. Thus, to match this, high temporal resolution is
often considered a requirement of intra-site spatial analyses
(Alperson-Afil 2017). In contrast, aside from a small minority of
possible exceptions (e.g. Leroi-Gourhan 1984; Audouze and Enloe 1997;
Alperson-Afil et al. 2009), the majority of Paleolithic sites are
palimpsests formed from sequentially overprinted occupations spanning
hundreds, or sometimes, thousands of years (Dibble et al. 1997; Bailey
and Galanidou 2009; Henry 2012; Vaquero, Chacón, et al. 2012). Given the
destructive nature of time-averaging on signatures of singular behaviors
(Stern et al. 1993; Stern 1994; Yellen 1977), much research has been
devoted to obtaining high-resolution data from aggregate assemblages
(Mallol and Hernández 2016; Vaquero and Pastó 2001). Protocols involving
a variety of sophisticated excavation and analytical methods including
\emph{décapage}, high-resolution 3D mapping, refitting, and microbiology
have been implemented in an attempt to extract individual occupations
from palimpsests (e.g. Leroi-Gourhan 1984; Bailey 2007; Goldberg and
Berna 2010; Aldeias et al. 2012; Henry 2012; Bisson et al. 2014; Mallol
and Hernández 2016). The application of these methods has been met with
variable success. It is still unclear what constitutes a brief instance
of activity or an episode of occupation (Vaquero, Alonso, et al. 2012;
Mallol and Hernández 2016), and when archaeological assemblages or
features can be isolated in time, determining their synchrony with the
rest of the assemblage remains problematic (Aldeias et al. 2012; Mallol
and Hernández 2016). Even thin assemblages from open-air sites, where
the structure of the archaeological assemblage appears to reflect an
isochronous event, can still represent several episodes of occupation
spread across considerable time (Bailey 2007; Bargalló, Gabucio, and
Rivals 2016; Roda Gilabert, Martínez-Moreno, and Torcal 2016). Rather
than extract individual activities from variably time-averaged deposits,
spatial methods have further exposed the underlying complexity of the
formation of the archaeological record. From the perspective of the
activities facies model, it seems as if palimpsests are not amenable for
understanding human behavior at this level. Yet, palimpsest sites are
far more abundant than the rare the examples when brief moments in time
are preserved and for this reason should remain the subject of
archaeological investigation.

This argument is not a recent realization. In response to Schiffer
(1975), Binford (1981) argued that researchers should not shoehorn the
archaeological record into the questions we ask but rather tailor our
questions to match the nature of the record itself. Since the 1970s,
archaeologists have also suggested that there are merits to
understanding behavior in time-averaged datasets (Foley 1981; Ebert
1992; Pettitt 1997; Bamforth, Becker, and Hudson 2005; Holdaway and
Wandsnider 2008; Bailey and Galanidou 2009; Clark 2017). Palimpsests are
the result of repeated occupations, often over multiple generations, and
thus afford the opportunity to investigate processes that structure
human behavior more in the long-term (Foley 1981; Bailey 2007;
Wandsnider 2008). The behavioral signatures preserved in these
temporally coarse assemblages likely tell us something about the
behavioral repertoire of populations or even taxa (Pettitt 1997).
Spatial patterning on this coarsened scale no longer reflects individual
activities but instead shows how broader, more temporally stable
(e.g.~the layout of a cave or rock shelter), external factors structured
behaviors. Patterns at this scale potentially document the interaction
between past populations and their physical environment (Brooks and
Yellen 1987; Bamforth, Becker, and Hudson 2005). In this regard, caves
are interesting because they impose physical constraints on how space
can be used. For instance, the dripline defines an interior, sheltered
portion of the space versus the exterior, open portion (Riel-Salvatore
et al. 2013). The movement of air and ventilation may have influenced
the placement of fires so as to avoid filling the space with smoke or
burning the fire too quickly. Subtle factors such as differences in
lighting or temperature across a cave may have influenced the structure
of behavior. At a more general level, when locations like caves are
subjected to repeated occupation over long periods of time, the residues
of previous occupations may structure subsequent ones (Bailey and
Galanidou 2009; Malinsky-Buller, Hovers, and Marder 2011). If these
factors structured behavior spatially, then it is likely that artifact
discard patterns may have been structured as well. Few studies have
applied intra-site spatial analyses to palimpsests as a whole, but
previous research has yielded interesting results. Bamforth et al.
(2005) and Bailey and Galanidou (2009), for instance, were able to
demonstrate the repeated discard of artifacts in the same locations over
extended periods of time. However, though these studies suggest that
palimpsests are structured (to some degree), further assessments of this
notion may require new methods that consider the behavioral inputs and
site formation processes under which palimpsests form.

\textbf{Spatial Analysis and Time-Averaged Assemblages}

Prior studies have often focused on describing spatial structure by
looking at variation in artifact density across a level or horizon (e.g
Baxter, Beardah, and Wright 1997; Baales 2001; Alperson-Afil 2008;
Aldeias et al. 2012; Gopher et al. 2016). Artifact density has been
argued to be advantageous for analyzing time-averaged deposits because
it is a function of discard. Simply put, areas where past people spent
more time will have accumulated more artifacts than other areas (Foley
1981). Thus, spatially delimited densities (i.e.~clusters) of
archaeological materials provide insight into how past behavior
structured the formation of the archaeological record (e.g Carr 1984;
Baales 2001; Yvorra 2003; Gopher et al. 2016). To demonstrate spatial
variation in artifact density, point and kernel density functions have
been widely applied (e.g. Baxter, Beardah, and Wright 1997;
Alperson-Afil et al. 2009; Aldeias et al. 2012; Alperson-Afil 2017).
However, ethnographic work shows that hunter-gatherer use of space is
often variable and unconstrained (O'Connell, Hawkes, and Jones 1991;
Clark 2017). Circumstantial factors such as weather, where individuals
are coming from, group size, kinship, and anticipated next destinations
influence the organization of any single episode of occupation
(Bamforth, Becker, and Hudson 2005). Moreover, differences in the
use-life of stone tools will also result in the differential
accumulation of artifacts (Shott 1998). Over time the relationship
between where behavior occurred and the artifact density will eventually
be lost. Most ethnographic examples come from open air contexts where
space is largely unconstrained. If, however, the frequency that
behaviors occur in a given location is influenced by the space itself,
as it likely is in caves, then the material traces associated with those
behaviors will appear in greater proportion to others. In a palimpsest,
instead of describing behavioral patterns in terms of artifact discard
intensity, they should be described in terms of artifact discard
likelihood. Approaching palimpsests in this manner recognizes the
frenetic conditions under which time-averaged assemblages form but also
considers that external factors that may more broadly govern the spatial
organization of activities.

Continuous or repeated occupation, particularly in context with low
sedimentation rates, can also introduce taphonomic biases into density
measures. Trampling of previously discarded and still exposed artifacts
will increase breakage within an archaeological assemblage (Lin et al.
2016). While a measure of trampling may be a useful indicator of
activity intensity (Reynard and Henshilwood 2018), unless breakage is
controlled for it will also inflate artifact density estimates and thus
must be considered when measuring spatial patterns in artifact discard.
Heating of surface or near surface artifacts also causes breakage and
makes artifacts more susceptible to subsequent breakage through other
processes (Mentzer 2009; Sandrine et al. 2005) thereby also inflating
density values. Without considering the impact of these processes, we
run the risk of interpreting density measurements directly as behavior
when in fact they reflect only taphonomic or geological processes or
some unknown mix of behavior and these other processes.

With these perspectives in mind, this study further examines whether
meaningful spatial patterning can be extracted from time-averaged
deposits. Our investigation focuses on data from the recent
re-excavation of the Neandertal cave site of Roc de Marsal, located in
southwest France. We develop methodologies specifically designed for
palimpsests using spatial statistics and behaviorally meaningful indices
developed from stone tool analysis. These indices are calculated as
ratios of different artifact classes and conditions. We then analyze the
spatial distribution of lithic material as the relative probability of
discard, while also controlling for the impact of breakage on an
archaeological assemblage. A neighborhood analysis is used to estimate
the spatial variability across each layer. While not all of our measures
showed spatial patterning, many of them did, suggesting that human
behavior at Roc de Marsal was structured over long time-scales. The
scale, variation, and co-variation of time allows for a discussion of
the dynamic nature of human behavior and its impact and signature in
palimpsests.

\textbf{Materials: The Site of Roc de Marsal}

Roc de Marsal (hereafter also RDM) is a small (\textasciitilde{}80
m\textsuperscript{2}) cave (Figure 1) situated approximately 5 km
southwest of Les Eyzies, southwest France, in a small tributary valley
of the Vézère River. The cave opening faces south-southwest, is about 80
m above the valley floor, and is just under the overlying plateau. The
site was intensively excavated twice in the last 50 years, first by Jean
Lafille from 1953 to 1971 and then more recently, from 2004 to 2010, by
a large collaborative team (Bordes and Lafille 1962; Turq 1979; Turq et
al. 2008; D. M. Sandgathe, Dibble, Goldberg, and McPherron 2011; Aldeias
et al. 2012). We know from Lafille's notebooks that he opened a meter
square in the central part of the cave and then expanded the excavation
into adjacent squares before switching to a strategy of excavating a
trench from approximately the entrance of the cave through to the back
(Goldberg et al. 2013) (see Figure 1). In addition, Lafille excavated a
connecting trench into a lateral extension of the main cave.
Unfortunately, Lafille did not record the spatial coordinates for
individual finds systematically enough to produce the type of dataset
required for our analysis. The more recent excavations took a new sample
from the site by pushing the west profile of Lafille's trench back one
meter along nearly its entire length, meaning from just outside the
current dripline to the back of the cave (see Figure 1). The dataset
from this excavation is what is analyzed here.

\textbf{Figure 1. The cave of Roc de Marsal (upper left), a map showing
the extent of the cave, the previous excavations, and the most recent
excavations (right), some of the stacked hearth features in Layer 9
(center left), and a panoramic (distorted) view of the west profile
towards the end of the new excavations. The main area analyzed in here
is shown in this latter photo (primarily E18 to K16).}

The new excavations recognized 13 layers; however, the main artifact
bearing layers are, from bottom to top, 9 through 2. The archaeology of
all of these layers is Middle Paleolithic (Turq 1985; Turq et al. 2008;
D. M. Sandgathe, Dibble, Goldberg, and McPherron 2011; Aldeias et al.
2012). The base (Layers 9-5) of the sequence is characterized by
Levallois technology. Layers 4-2 represent a switch to Quina techniques
of blank production. Denticulates are numerous in Layers 9-7, and the
proportion of scrapers increases throughout the sequence. Layers 9-7 are
characterized mostly by red deer and roe deer (Hodgkins et al. 2016;
Castel et al. 2017). Reindeer are present in Layers 9-8 and increase in
Layer 7 and above. Bovines and horse are also present. By Layer 5 times
reindeer dominate the assemblage, accounting for approximately 70\% of
the NISP, with the rest consisting mainly of horse and bovines. This
trend continues in Layer 4 where reindeer reach over 80\% of the
assemblage NISP. Layer 3 shows a more diverse faunal spectrum (though
sample sizes are small), and Layer 2 shows a return to the Layer 4
pattern. The sequence has been intensively dated using a variety of
thermoluminescence and optically stimulated luminescence methods
(Guibert, Lahaye, and Bechtel 2009; Guérin et al. 2012; Guérin et al.
2017). The different dating techniques provide somewhat different
results, but Guérin et al. (2017) use their most reliable set of ages to
place the base of the sequence in MIS 4 and that the top of the sequence
in MIS 3. In their view, the transition between MIS 4 and 3 likely
occurs sometime during the deposition of Layers 6 and 5 . How to
reconcile this age model with sharply contrasting paleoenvironmental
proxies coming from geological observations and the faunal spectrum is
as yet unresolved.

A total station was used to record the location of all artifacts larger
than 25 mm (piece provenienced finds) (see also Sandgathe et al. (2018)
for same methods applied to Pech de l'Azé IV). Complete teeth and
complete bones of non-microfaunal remains smaller than 25 mm were also
recorded with the total station. All other small artifacts (\textless{}
25 mm) were included with the sediment buckets and retrieved after wet
screening as bulk samples (small finds). At the time of our study, the
complete lithic analysis database was available, but only a limited
sample of the fauna had been studied, and so these data could not be
used here. A summary of the data set we use is provided in Table 1. Note
that Layers 6 and 3 were not included in the spatial analysis because
their horizontal extent is quite limited. Layer 2 was not included
because its sample size is small. All of the other layers could be
traced over a large extent of the cave, though the spatial extent of
each layer does vary somewhat, and they all have large sample sizes.
However, there are differences in artifact density. Layer 9 has the
highest number of lithics per liter (9.4), Layers 8 and 7 have similar
densities (7.3 and 8.0 respectively), Layer 5 shows an intermediate
value (4.5) and lithic densities are substantially lower in Layer 4
(1.5).

\begin{longtable}[]{@{}lrrrrrr@{}}
\caption{Summary of liters excavated and lithic counts for the Roc de
Marsal layers analyzed here. The counts include first all lithics and
then breakdown by cores, tools, flakes and burned
lithics.}\tabularnewline
\toprule
Layer & Liters & Lithics & Cores & Tools & Flakes &
Burned\tabularnewline
\midrule
\endfirsthead
\toprule
Layer & Liters & Lithics & Cores & Tools & Flakes &
Burned\tabularnewline
\midrule
\endhead
4 & 1925 & 2932 & 55 & 465 & 1767 & 61\tabularnewline
5 & 532 & 2411 & 82 & 279 & 1781 & 205\tabularnewline
7 & 504 & 4025 & 160 & 244 & 3113 & 833\tabularnewline
8 & 511 & 3709 & 143 & 199 & 2824 & 764\tabularnewline
9 & 868 & 8132 & 216 & 276 & 6408 & 3150\tabularnewline
\bottomrule
\end{longtable}

Intact hearths are also present, particularly in Layers 9 and 7 (Aldeias
et al. 2012; Goldberg et al. 2012). These hearths vary in size from as
small as 30 x 50 cm to roughly a meter on each side and are in many
cases composed of the classic sequence of reddened sediment,
charcoal-rich base, and overlying ash. There are multiple instances in
which stacked hearths could be observed as well (see Figure 1). As
measured by the frequency of heated lithics, Layer 9 shows the most
intense use of fire (D. M. Sandgathe, Dibble, Goldberg, McPherron, et
al. 2011). Thereafter the frequency of burned lithics decreases in the
sequence with a near absence of evidence of fire (in the form of visible
hearth features or in the form of heated bones or lithics) in Layers
4-2. Roc de Marsal is interesting for our spatial study in part because
it highlights issues with palimpsest dissection. In an attempt to
understand spatial patterning between fires and artifact discard,
previous research at Roc de Marsal used a combination of
micromorphological approaches and horizontal excavation (or
\emph{décapage}) to establish potential surfaces representing brief
episodes of activity associated with combustion features. However, the
degree of synchrony between lithic assemblages and other archaeological
features could not be confidently established. Subsequent spatial
analyses failed to reveal any meaningful patterning between the density
of lithics and combustion features within each \emph{décapage} surface
(Aldeias et al. 2012).

Rather than attempt to increasingly resolve temporal units of analysis,
as was previously done (e.g. Aldeias et al. (2012)), this study searches
for behavioral patterns in the palimpsests that comprise the
archaeological horizons at Roc de Marsal. Thus, each layer was spatially
analyzed as a single unit of analysis. One of the disadvantages of
looking for spatial patterning at Roc de Marsal is that the horizontal
extent of excavation in each layer is limited. While the excavations
sampled from the front to the back of the cave, the width of the
excavation was generally only about 1 m. Further, this excavation
corridor went mostly through the central portion of the cave, and so we
have very limited samples from near the cave wall. We note, however,
that Roc de Marsal is not a large cave, and so the excavations actually
sampled a large proportion of the space inside the cave. In addition,
from a practical point of view, it would have been inadvisable to
excavate laterally more of the cave (i.e.~up to the cave wall) as it
would have permanently removed a record of the stratigraphic sequence
there. Thus, while we miss Lafille's excavated sample in our analysis,
the sample we were able to analyze is not atypical for this kind of
site. On the other hand, while the small size of the Roc de Marsal cave
means that a rather limited excavation captures a large percentage of
each layer, it also means that where behaviors could have taken place in
the cave was also constrained, and this could, in turn, force a certain
amount of spatial association.

\textbf{Methods: Neighborhood Analysis}

This study employs a neighbor analysis to characterize spatial variation
in artifact discard patterns across each layer. Moving windows or
neighborhood analyses are used in geographic information science to
document local variation in a given variable (Lloyd 2006). Neighborhoods
of a defined two-dimensional size and shape are drawn over the area of
interest. Observations located within each neighborhood are summarized
according to a given statistic (Hagen-Zanker 2016). The end result
produces a map where each data point represents the calculation of that
statistic for its neighborhood. Often times such techniques produce
smoothed rasterized representation of the moving window result but they
can also be point based. Point density functions are a common
application of this type of analysis in archaeology and have been
applied at a variety of different scales (Baxter, Beardah, and Wright
1997; Abe et al. 2010; Alperson-Afil and Goren-Inbar 2010; Aldeias et
al. 2012; Blasco et al. 2016).

Here we use a circular window of 30 cm in radius as our neighborhood and
calculate the local variation of each artifact metric (see the
calculation of metrics below) based on this sample. This window size was
determined largely by the spatial scope of the study area. Too small a
window risks introducing local noise into the result, and too large a
window risks aggregating meaningful local variation into noise. While
GIS software packages offer a variety of moving window tools (e.g.~Point
density, Kernel density, Optimized hotspot analyses) (McCoy and Johnston
2001), our study requires the flexibility to calculate various
artifact-metrics. Thus we custom developed the moving window analysis
used here in R with the ``rgeos'' package specifically for the purpose
of calculating each metric. We have included the raw data, source code,
and the markdown document used to create this publication (cf. Clarkson
et al. (2015); Marwick (2017); McPherron (2018)).

\textbf{Methods: Calculation of Metrics}

We use seven indices to describe discard patterns of the following
artifact classes and conditions: (1) overall stone artifact density, (2)
complete flake weight, (3) cortex to mass ratio, (4) proximal flake
ratio, (5) burning ratio, (6) scraper to flake ratio (7) core to flake
ratio. The mechanics and behavioral justification for these ratios are
as follows.

\emph{Artifact density} for each neighborhood was calculated as the
number of stone artifacts. Though density measurements are potentially
biased by breakage, this describes the general distribution of artifacts
across each layer and, therefore, provides a baseline comparison for the
other indices.

\emph{Complete flake weight} is calculated as the median weight of
complete flakes within a neighborhood. The median flake weight in each
neighborhood (See moving window section) is then divided by the median
flake weight of the entire layer. This allows the values to be
standardized such that flake weight is comparable across layers. This is
a common practice in spatial analysis, as it allows one to find local
variation in a metric given its global value (Isserman 1977). Values
associated with this metric center on 1. Values above 1 mean that the
median flake weight in the window is greater than median flake weight of
the whole layer and values below 1 mean the opposite. Our expectation
for this measure is that areas of tool production (i.e.~retouch) and
rejuvenation might have smaller flakes, though the sensitivity of this
measure for these behaviors is tempered in this case by our 2.5 cm size
cut-off.

\emph{Cortex to mass ratio} can act as a proxy for reduction stage under
the assumption that greater proportions of the cortex are removed at
early stages of the production sequence (Oron and Goren-Inbar 2014);
areas associated with early-stage reduction will have higher cortex to
mass ratios where middle and later stages will have less (Toth 1985).
This assumption, however, is in part impacted by the technology
(e.g.~Levallois versus Quina techniques for blank production) and by
lithic transport patterns (see below), neither of which was constant
throughout the sequence. Some caution is then required when considering
differences between layers. To calculate this ratio, for each artifact,
the relative proportion of cortex was previously measured using
intervals and here converted to a decimal based on the interval
mid-point. Surface area is estimated as a two-dimensional rectangular
surface by multiplying the artifact length by width (Douglass et al.
2015). Estimates of the cortical surface area are then derived by
multiplying the proportion of cortex by the surface area. Cortical
surface area estimates are then divided by mass to control for the
influence of size on the cortical surface. Aside from reduction
intensity considerations, the movement of already worked materials into
and out of the site will affect the overall amount of cortex in an
assemblage. Based on the cortex ratio (Dibble et al. 2005), it has
already been established (Lin, McPherron, and Dibble 2015) that in
Layers 5 and 4 cortex is under-represented, whereas in Layers 9-7 the
expected amount of cortex is present. Nevertheless, here we can examine
how the cortex that is present is distributed spatially.

\emph{The proximal flake ratio} is measured by calculating the ratio of
proximal flakes to complete and proximal flakes. Proximal flakes here
are defined as broken flakes preserving more than 50\% of the platform.
As this index increases, the representation of proximal flakes within a
given neighborhood increases meaning a greater degree of breakage. Thus,
high values reflect an abundance of breakage and low values the
opposite. We included this breakage index to infer areas of intense
trampling. Holding other factors constant, areas that are subject to
more activity or trampling should have a higher frequency of flake
breakage (Nielsen 1991). However, breakage rates will also depend on
factors such as the substrate and the size/morphology of the flakes
themselves, both of which could vary across the layer and
size/morphology will vary between layers for a number of reasons
including changes in blank production technology. The relationship
between flake size/morphology and breakage is poorly understood, and so
we are unable to control either of these factors here. We do know,
however, that burning increases breakage and so heated flakes are
removed from this calculation.

\emph{The burning ratio} is designed to investigate the spatial
structure of burned artifacts as a proxy for the placement of fires.
Fire placement within the cave was likely structured and fires may have
in turn structured where behaviors took place (e.g Galanidou 1997;
Pettitt 1997; Gopher et al. 2016), thus understanding whether there is
structure in the spatial distribution of burned artifacts may provide
insight into the organization of behaviors in space. Note that we are
not suggesting that artifacts were intentionally heated, rather they are
incidentally heated when fires are constructed on artifact bearing
deposits or tossed directly into an active fire (Aldeias et al. 2016).
Thus if there is consistency in where burning events occurred, then
there should also be structure in the burned artifacts. However, burning
increases breakage which then over-represents burned lithics relative to
other lithics within any given neighborhood. Rather than computing the
relative frequency of burned lithics to all other lithics, we account
for breakage by only considering complete and proximal flakes in our
estimation. The ratio of burned complete flakes and burned proximal
flakes to all complete and proximal flakes was calculated for each
neighborhood.

\emph{The scraper to flake ratio} and \emph{core to flake ratio} are
designed to characterize the discard of cores and scrapers relative to
flakes across each layer. Previous studies have argued that artifact
types such as scrapers and cores are also structured around combustion
features (Pettitt 1997), and so we may expect that locations subject to
continuous burning over long time scales may also structure the discard
patterns of scrapers and cores. The distribution of scrapers is
calculated as the relative frequency of scrapers to the number of
proximal and complete flakes. As with scrapers, the spatial distribution
of cores is calculated as the frequency of cores relative to the number
of proximal and complete flakes (core to flake ratio). The inclusion of
complete and proximal flakes only is, again, to account for potential
distortion of values due to breakage introduced by taphonomic process.

\emph{Local Indicators of Spatial Association}

The neighbor analysis allows us to observe spatial variation in the
calculated metrics. However, it provides no way to quantitatively
characterize the meaningfulness of the variation we observe in each
layer. To address this, a Local Moran's I test is applied to the results
of the moving window analysis (Anselin 1995). Global Moran's I then
characterizes the spatial structure of a given attribute as non-random
clustered, non-random even, or random (Bivand, Pebesma, and Gómez-Rubio
2013). This is done using a neighborhood analysis to analyze first local
relationships between values observed at a specific point and its
neighbors. The global structure of a distribution of values is then
calculated based on these measurements. Since Moran's I requires an
understanding of spatial relationships at a local scale, it can be
broken down into its individual parts. Statistical tests can then be
applied to each of these parts to determine the statistical significance
of clustering or dispersion at the local scale (Anselin 1995; Lloyd
2006; Bivand, Pebesma, and Gómez-Rubio 2013). The significance of
clustering around each point in these analyses can be plotted in space.
Using this method, the statistical meaningfulness of spatial variation
in each metric is detected as statistically non-random ``clusters'' on
the local scale. Results of the Local Moran's I test can then be used to
demarcate areas of statistically significant clusters of high and low
values and outliers (Lloyd 2006). Outliers are defined as statistically
significant clusters of high values that are surrounded by clusters of
low values or vice versa. This technique is beneficial as it provides an
objective determination of when spatial variation in each calculated
metric is the result of non-random processes. In doing so, this provides
a way to better discern between spurious variation and systematic
structuring of artifact discard patterns.

This study uses the spdep package in R to implement Local Moran's I
methods as developed by Anselin Anselin (1995). Clusters of high and low
values were determined with 95\% confidence. Hereafter, statistically
significant clusters of high values are referred to as high clusters and
statistically significant clusters of low values will be referred to as
low clusters. Outliers, where non-random high values are surrounded by
low values or the opposite, are referred to as high-outliers and
low-outliers respectively. An additional note of importance is that both
Local and Global Moran's I are scale-dependent and require the
definition of a neighborhood size. Since the results of this test are
directly influenced by the size of the neighborhood used, additional
measures must be taken to determine the most appropriate window size
(this window is independent of the moving window used to calculate the
above metrics at each artifact location). We used a correlogram to
estimate the most appropriate window size for each metric. This approach
involves calculating Global Moran's I at a series of incrementally
increasing neighborhood sizes (Bivand, Pebesma, and Gómez-Rubio 2013).
The results are then plotted against neighborhood size. The neighborhood
size exhibiting the greatest autocorrelation (Moran's I) was then chosen
as the window size for the Local Moran's I test (see SOM). Clusters of
statistically significant high and low values were then used to further
explore the significance of the spatial variation of each metric.

\emph{Interpretative Framework}

One large risk of the approach taken here is that, given the number of
indices (metrics) examined and the nature of the spatial statistics
applied to them, any one layer from any site is likely to produce a
pattern of some kind. While \emph{a priori} we expect caves to produce
patterns for the reasons outlined in the introduction, interpreting
these patterns in the absence of any specific \emph{a priori}
predictions or expectations about what patterns, in particular, are
expected risks falling into the Texas sharpshooter fallacy wherein
patterns are searched for, found and only then are explanations built to
account for them. To help mitigate this problem, we look for repetition
in the patterning between layers. In this framework, each layer is
viewed as an independently drawn sample of the accumulation of behaviors
preserved at the site with the one (mostly) constant being the physical
configuration of the cave. Patterning that repeats across these
independent samples is then more robust and likewise are any
explanations for this patterning. The extent to which layers actually
represent independent samples can be challenged on the basis, for
instances, that visible traces of previous activities (in the underlying
layer) may continue to structure behavior in the new layer just as we
suspect happens within a layer. We see no clear way of addressing this
issue in a single cave sequence. Clearly, strong behavioral inferences
require consistent patterning in multiple sites.

\textbf{Figure 2. The spatial distribution of lithic artifact density
across each layer. Deep red colors indicate areas with high densities of
artifacts whereas lighter red values grading to blue indicate lower
artifact densities. The coloring for each layer is scaled separately,
and the values are per 30 cm\textsuperscript{2}. The dashed line
represents the current dripline (see also Figure 1).}

\textbf{Results: Inter-level comparisons of Spatial Organization}

\emph{Density}

Layers 5 and 4 are less dense than the underlying Layers 9, 8 and 7
(Table 1), and the spatial distribution of artifact density is variable
between layers (Figure 2). At the base of the sequence, in Layer 9,
artifact density is predominately concentrated towards the front of the
cave, though a smaller concentration of artifacts is also observed at
the area beyond the cave terrace. Artifact densities in Layer 8 are also
primarily concentrated at the front of the cave. Additionally, moderate
concentrations of artifacts are present throughout the cave terrace.
Layer 7 shows spatially discrete concentrations of high artifact
densities occurring in the front and terrace parts of the cave. In Layer
5 areas of high density are spread throughout the cave terrace. A small
concentration of artifacts also occurs in the front of the cave.
Finally, in Layer 4 the highest artifact densities are again towards the
front of the cave. Note that consistent with Aldeias et al. (2012), the
locations of visible fire features in Layers 9, 7, and 5 do not appear
to be related to artifact density.

\textbf{Figure 3. Clustering in median flake weight values across each
layer. Orange represents statistically significant clusters where median
flake weight is high. Blue represents statistically significant clusters
where median flake weight is low. Note that only complete flakes are
included. Grey represents non-significance. The dashed line represents
the current dripline (see also Figure 1).}

\emph{Complete Flake Weight}

Though median flake weight varies in magnitude between layers, there is
consistency in its representation across the excavated areas of the cave
in Layers 9, 8, 7, 5 (Figure 3). In each of these layers, the median
flake weight is greatest in the front of the cave. However, there is a
considerable degree of noise in this pattern (SOM Figure 6). Despite the
noise, the Local Moran's I results reveal that statistically significant
high clusters of median flake weight values are consistently found in
the front part of the cave. Moreover, statistically significant clusters
of low median flake weight values occur towards the cave terrace. A
linear regression reveals a significant relationship showing that flakes
greater than the median flake weight for each layer become increasingly
over-represented as one moves from the cave terrace to the front of the
cave (p-value: \textless{} .001, R-squared: 0.115). Layer 4 shows a
different pattern with no relationship between the distance from the
front of the cave and median artifact weight.

\textbf{Figure 4. Spatial distribution of non-random clusters of the
cortex to mass ratio. Orange represents significant clusters of the high
cortex to mass ratio. These are areas where there are greater amounts of
cortex. Blue represents areas with significant clusters of the low
cortex to mass ratio meaning very little cortex is found in these areas.
The dashed line represents the current dripline (see also Figure 1).}

\emph{Cortex}

Non-random clustering of the cortex to mass ratio is observed in each
layer (Figure 4). However, there is little consistency in the structure
of the patterning between layers (SOM Figure 7). Layer 9 demonstrates
the greatest degree of clustering. Large statistically high clusters of
cortex to mass ratios occur on the cave terrace. The greatest clustering
occurs nearest to the dripline. However, smaller high clusters are also
present on the part of the cave terrace farthest from the mouth of the
cave. Low clusters are at the farthest most extreme of the cave terrace
as well as at the front of the cave. High and low clusters in Layer 8
are situated throughout the cave. Layer 7 shows no coherent patterning
in the locations of both high clusters and low clusters as both types of
clusters occur throughout all parts of the cave. In Layer 5 clusters
representing low cortex to mass ratios occur predominantly on the cave
terrace, farthest away from the dripline and mouth of the cave. In Layer
4, high clusters are in the front of the cave and at the very front of
the cave terrace and clusters with low cortex are in the center portion
of the cave terrace.

\textbf{Figure 5. Top: The spatial distribution of burned flakes in
relation to all flakes. Note how layers 7 and 9 show the greatest degree
of spatial organization of burned flake proportions. Bottom: The spatial
distribution of high and low clusters associated with the burning ratio.
Orange indicates areas with statistically non-random high proportions of
burned flakes. Blue represents statistically significant areas where
there are fewer burned flakes. The dashed line represents the current
dripline (see also Figure 1).}

\emph{Burned Flakes}

The lower Layers 9, 8, and 7 all possess significantly higher
proportions of burned lithics than in upper Layers 5 and 4 (Table 1).
Layer 9 exhibits the greatest amount of structure (Figure 5). High
proportions of burned flakes form a single non-random high cluster that
straddles the dripline. The extent of the cluster begins in the front of
the cave and extends on to the cave terrace. The spatial patterning of
burned lithic values encompassed by the extent of this cluster creates a
focal point comprised of the highest proportion of burned flakes. The
proportion of burned flakes systematically decreases from this focal
point. Non-random low clusters of burned flake proportions surround the
edges of this focal point and extend toward the front and central parts
of the cave as well as on to the farthest most reaches of the cave
terrace. Despite the absence of visible fire features, Layer 8 has a
proportion of burned lithics intermediate to Layer 9 below it and Layer
7 above it. The majority of Layer 8 has low proportions of burned
flakes. Low clusters of burned flake proportions are, thus, found
throughout all sections of the cave in this layer. A few high
concentrations of burned flake proportions form along the right side of
the front portion of the cave and cave terrace. While the overall
proportion of burned flakes is considerably less, Layer 7 exhibits
spatial patterning similar to that of Layer 9. High proportions of
burned flakes form two non-random clusters in the front and terrace
sections of the cave. Burned flake proportions surrounding these loci
are largely not significant or comprise statistically significant low
clusters of burned flakes. In Layer 5 the overall percentage of burned
flakes is relatively low; however, some clustering of burned flake
proportions can be observed. The highest proportions of burned flakes
form a statistically significant cluster on the terrace part of the
cave. Although Layers 9, 7 and 5 all have evidence of combustion
features, no single feature directly corresponds to areas of high
burning. Layer 4 has the lowest number of burned lithics. Variation in
the proportion burned flakes is observed, however, and 75\% of the
values are less than .05. Though the Local Moran's I test revealed both
high and low clusters of burned lithics, there is no structure to their
overall distribution.

\textbf{Figure 6. The spatial distribution of high and low clusters
associated with the ratio of proximal flakes to all flakes. Orange
indicates statistically high clusters of proximal flake ratios, meaning
there are more broken flakes relative to complete flakes. Blue indicates
statistically low clusters of proximal flake ratio values. In these
areas there are fewer broken flakes and more complete flakes. The dashed
line represents the current dripline (see also Figure 1).}

\emph{Proximal Flake Ratio}

The distribution of flake breakage is variable across layers (SOM Figure
8). In Layer 9, low clusters of proximal flakes relative to all flakes
are situated on the cave terrace. Clustering showing statistically
significant high proportions of broken flakes are located in the front
section of the cave (Figure 6). The distribution of both the high and
low clusters of broken flakes also inversely corresponds with the
spatial distribution of burned flakes. This high cluster corresponds
with the statistically significant low cluster of burned lithics in the
front of the cave but is situated on the edge of a high cluster of
burned flakes, whereas low clusters representing low proportions of
proximal flakes are directly within the burn zone. In Layer 8 high and
low clusters of proximal flake proportions are distributed throughout
the layer with no general patterning to their distribution. The relative
frequency of proximal flakes in Layer 7 is structured similarly to Layer
9. Clusters representing high breakage patterns are situated toward the
front of the cave whereas clusters of low breakage are found in and
extending beyond the cave terrace away from the mouth of the cave. The
high cluster in the back also exhibits a similar spatial association
with the focal points of burned lithics. High clustering indicative of
breakage, corresponds with the low clusters of burned flakes and
vice-versa. For Layer 5, clusters with low proportions of proximal
flakes are found on the cave terrace whereas high clusters are in the
front of the cave and in the parts of the excavation farthest from the
mouth of the cave. Finally, this index in Layer 4 is generally not
structured in space as much of the variation is non-significant. The
high and low clusters that are present show little evidence of spatial
patterning.

\textbf{Figure 7. The spatial distribution of high and low clusters of
the relative frequency of scrapers in comparison to flakes. Orange
points reflect areas where the proportions of scrapers are high. Blue
reflects statistically significant areas where scraper proportions are
low. The dashed line represents the current dripline (see also Figure
1).}

\emph{Scraper to flake ratio}

There is little consistency in how the proportion of scrapers in
comparison to flakes is distributed throughout each layer (SOM Figure
9). Overall, scrapers comprise a small proportion of each assemblage.
However, non-random structure is observed in each layer (Figure 7). In
Layer 9, the high scraper to flake ratio values are primarily in the
front of the cave and at the outer edge of the terrace. Though noisy,
statistically significant clusters of the high scraper to flake ratio
values appear to form a ``ring-like'' shape that surrounds a
statistically significant cluster with low values. This low cluster also
corresponds to the area where the proportion of burned flakes is the
highest. In Layer 8, the largest high cluster of the scraper to flake
ratio values are in the front of the cave along the drip line. However,
high and low clusters are found throughout the cave. Within Layer 7 high
clusters are in the front of the cave whereas low clusters are
predominantly situated on the cave terrace. A high cluster also appears
to correspond with the cluster of high proportions of burned flakes,
with a large degree of overlap. Additionally, a few isolated small high
clusters can be observed in the middle of the cave as well. The majority
of the clustering of high scraper values in Layer 5 occurs in the front
whereas low-clustering occurs toward the front of the cave terrace and
beyond. Scraper to flake ratios within the cave terrace of Layer 5 is
predominantly non-significant. Layer 4 shows the greatest degree of
spatial structure with high clusters occupying the cave terrace and
beyond whereas the low clusters in the front and central sections of the
cave.

\textbf{Figure 8. The spatial distribution of the core to flake ratio.
Orange points reflect areas where the proportions of cores are high.
Blue reflects statistically significant areas where core proportions are
low. The dashed line represents the current dripline (see also Figure
1).}

\emph{Core to flake ratio}

The core to flake ratio shows a fair degree of consistency across the
layers (SOM Figure 10). Greater proportions of cores (low flake to core
ratio) occur in the front of the cave and on the cave terrace. In
addition, high clusters of the core to flake ratio occur; however, this
pattern is not as systematic as the pattern observed in median weight.
The organization of cores within each layer also shows similarities with
the spatial distribution of scrapers. Much like the pattern observed in
the scrapers, the clustering of core to flake ratios shows the greatest
amount of patterning in Layer 9. Aside from a few very small high
clusters of cores in the front and on the cave terrace, high clusters of
the core to flake ratio are in the front of the cave (Figure 8). As with
the scraper to flake ratio, high clusters of the core to flake ratio are
situated on the edge of the area with high burned flake proportion
values. The large low cluster of core to flake ratios on the cave
terrace corresponds to the area of highest burned flake proportions. Low
clusters of cores extend beyond the cave terrace away from the drip line
to the very front of the excavation. High and low clusters of core to
flake ratios are detected in Layer 8 but do not appear to show any
apparent spatial structure. In Layer 7 statistically significant high
and low clusters of cores are relatively small compared to Layer 9. The
largest high clusters are on the cave terrace and front part of the
cave. Few smaller clusters of high proportions of core to flake ratios
are also found toward the central part of the cave and toward the very
front of the excavation. In Layer 5, there is a fair amount of spatial
separation between high clusters and low clusters of cores. High
clusters occupy the front and cave terrace nearest to the dripline
whereas low clusters are situated in the middle and form of the cave
terrace. In Layer 4, clusters of high core to flake ratios are in the
front of the cave and the cave terrace.

\textbf{Discussion}

The results yielded by this study make for several talking points
regarding the spatial structure of time-averaged assemblages as well as
for drawing behavioral inferences from them. The results of each index
show spatial variation across each layer. None of the variations in each
of these measurements tracks the overall density of artifacts,
suggesting that they provide a useful, if not better, means for
understanding stone artifact discard patterns preserved within
palimpsests. Moreover, the Local Moran's I showed that the spatial
variation of every metric also formed statistically non-random clusters.
In the literature repetition is a common theme in how time-averaged
patterns arise and, thus, are thought to represent the ``modal
behavior'' (Foley 1981). At RDM, median complete flake weight is always
greatest within the cave itself (as opposed to the terrace) in Layers
9-5. This result seems to suggest that at least some discard behavior
was consistently carried out in specific parts of the cave. Though size
sorting due to small changes in slope within the cave, particularly
towards the mouth but also some towards the back, also need to be
further examined. However, many of our indices showed inconsistent
variation in both their spatial distribution and in relationship to one
another across each layer. Despite the detection of non-random
patterning, this lack of consistency in patterning makes it difficult to
interpret many of these metrics in terms of long-term behavioral trends.
This may imply that most of these discard behaviors were not structured
by the morphology of the cave. Rather, it means that behavior was
structured in relation to the other behaviors being carried out in the
cave. The tendency for high core to flake ratios to cluster away from
high cortex to mass ratio values in Layers 9-5 may suggest that cores
were consistently discarded away from where they were initially reduced.
This seems to support the notion that long terms discard patterns
associated with the various activities carried out at Roc de Marsal did
not occur in a vacuum. Palimpsests form as the result of the interaction
of many different behavioral, environmental, and taphonomic processes
through time and space. Thus, the non-random structure identified by the
Local Moran's I test likely emerged from this dynamic interaction
instead of from any modal or singular behavior. In this sense, it does
not seem sensible to attempt to interpret these patterns in terms of the
singular behaviors assumed to be reflected in them.

This is no more apparent than in patterning surrounding burning episodes
at RDM. Despite the fact, that fire features span the excavated area in
both Layers 9 and 7, there is a high degree of structure in each level.
Counterintuitively, high proportions of burned flakes form large focal
points that generally do not correspond to any single combustion
feature. In the long term, aggregation localities subjected to burning
would likely have been maintained with ash raked out and redeposited (an
action detected micromorphologically in some cases at Roc de Marsal, see
Aldeias et al. (2012) and Goldberg et al. (2012)). If this is so, then
the weak correlation between visible combustion features in Layers 9 and
7 and the hotspots of heated lithics may be due to some extent to
maintenance over time, thus disassociating ash with the burn location.
This, in turn, may be due to a limited number of places that fires can
be placed. However, this was not true in all cases and thus it is
difficult to generalize whether heated artifacts are a better index of
fire locations than visible traces of fire.

Nevertheless, it is interesting that the larger area reflected of
continuous burning also has an effect on the discard patterns of other
artifact types. Clearest are the high core to flake ratio, scraper to
flake ratio and proximal flake ratio, which are all situated on the
edges or just outside of these burn zones (Figure 9). The propensity to
discard scrapers and cores, in combination with high levels of flake
breakage, suggests that areas close to these areas of high burned flake
ratios are high activity areas. Though it is not possible to know which
behaviors were carried out at this location, it is clear that fire
played an important enough role to structure the discard behaviors of
the occupants of Roc de Marsal over long timescales. Thus when we look
at the Layer 9 assemblage without reference to the spatial data,
scrapers less frequently show signs of heating in comparison to
unretouched flakes (Fisher Exact test, p = 0.035). This pattern is less
clear in Layer 7 than it is in Layer 9. Though proximal flake ratios are
very similar, the pattern with the core to flake and the scraper to
flake ratios is less clear. Still the pattern is strong enough that at
the assemblage level again scrapers are less likely to show signs of
heating (Fisher Exact test, p = 0.000). Why the spatial structure is
less clear in Layer 7 is difficult to say. Both Layers 9 and 7 have
evidence of the structured use of space in the form of stacked hearths
(Aldeias et al. 2012). However, Layer 7 also has more than half as many
lithics as Layer 9, and the areas immediately around the burning
hotspots in Layer 7 are less well sampled given the limits of the
excavation and some clandestine excavations that occurred before the
most recent excavations. By comparison, in Layer 8, which has a high
percentage of heated lithics but no visible evidence of hearths and no
clear spatial patterning in heated lithics, scrapers are just as likely
as flakes to show signs of heating (Fisher Exact test, p = 0.297). Thus
the spatial patterning observed in these Layers 9 and 7 provides insight
into how archaeological signatures of structured behavioral patterns
transform during palimpsest formation. Though high-resolution data can
show how the discard of formalized tools is associated with individual
fire features (Pettitt 1997), over time, it is not any one single
feature that structures the record but rather these broader zones of
continuous use and their spatial relationship with the cave.

\textbf{Figure 9. Each figure has as its base map the areas of high
burning in Layer 7 (top row) and Layer 9 (bottom row). Overlaid on this
are the high clusters of scrapers and of cores and the low clusters of
complete flakes. The dashed line represents the current dripline (see
also Figure 1).}

The patterns revealed at RDM demonstrate that the aggregation of
episodes of behavior over-time imposes structure on the time-averaged
archaeological record. This alone demonstrates the utility of
palimpsests for understanding hominin behavior over long time-scales.
However, interpreting these patterns and their evolutionary relevance is
currently difficult. The disconnect between individual instances or
relatively brief episodes of behavior and the long-term pattern present
at Roc de Marsal reflects a well-known phenomenon in social science
often referred to as emergence (Schelling 1978; Miller and Page 2007).
Though it has been documented in primate tool use studies (Luncz et al.
2016), it has been rarely discussed in archaeology. If time-averaged
behavioral patterns are better studied as a whole, instead of being
reduced to the individual parts, then we must reconsider how we approach
palimpsests. Rather than continue to attempt to isolate individual
occupation or behavioral episodes, it may be more useful to devise
research projects that attempt to better understand the complex
interactions of hominin behaviors, ecology, taphonomy, and the formation
of the archaeological record. Though this raises issues over how to
develop analogs to understand processes that operate on time-scales
beyond our lifetime, the combination of ethnoarchaeology and
computational individual-based modeling is well poised to do this.

Ultimately, Roc de Marsal provides some initial insights into the
behavioral structure of palimpsests. However, the notions asserted here
will require further validation through future research. This will
require the continued development of methods that provide robust
expectations for what aggregated patterns of behavior look like in the
archaeological record. With the increased use of spatial statistics and
agent-based modeling, it is becoming possible to generate and test
hypotheses surrounding the behavioral and taphonomic complexity that
underlay palimpsests.

\textbf{Conclusion}

Rather than dissect palimpsests, this study argues that there are merits
to analyzing palimpsests as a whole. In doing so, we develop a novel
methodology that combines some of the taphonomic and behavioral inputs
that influence the formation of time-average assemblages. The
application of these methods to the aggregate assemblages of Roc de
Marsal shows that palimpsests, at least in this case, contain structured
patterns. However, in many cases, these patterns are not easily tied
back to behavior. The structure revealed in the archaeological layers at
Roc de Marsal provides a starting point for discussions regarding the
nature of the patterns present within palimpsests and the ways in which
to best draw behavioral inferences from them. Ultimately, this will
require continued work and the further incorporation of sophisticated
geospatial statistics and computational modeling.

\textbf{Acknowledgments}

The research at Roc de Marsal had the financial support of the US
National Science Foundation (Grants \#09177739 and \#0551927), the
Leakey Foundation, the University of Pennsylvania Research Foundation,
the Service Régional de l'Archéologie d'Aquitaine and the Conseil
Général de la Dordogne. The authors thank Jean-Jacques Hublin and the
Max Planck Society for supporting this research presented here. JR
thanks David Braun and the Center for the Advanced Study of Human
Paleobiology at George Washington University for supporting his
research. The approach taken here to time-averaged assemblages
benefitted from valuable discussions with a number of people including
Simon Holdaway, Sam Lin, Željko Režek, and Luke Premo. A special thanks
goes to José Ramón Rabuñal Gayo who reviewed the code and code/text
consistency. As always, all mistakes remain our own. We note that Harold
Dibble participated fully in the research presented here and was able to
comment on the nearly final manuscript. The Roc de Marsal team misses
him greatly.

\textbf{Supporting Information}

S1. rMarkdown file used to make this document (Reeves et al. - RDM Time
Averaging.rmd).

S2. Supplementary information (Reeves et al. - RDM Time Averaging
SOM.pdf).

S3. rMarkdown file used to make the supplementary information (Reeves et
al. - RDM Time Averaging SOM.rmd)

S4. Data files needed to compile markdown documents (Reeves et al. - RDM
Time Averaging.zip)

\section*{References}\label{references}
\addcontentsline{toc}{section}{References}

\hypertarget{refs}{}
\hypertarget{ref-Abe2010}{}
Abe, Yoshiko, Curtis W Marean, Peter J Nilssen, Zelalem Assefa,
Elizabeth C Stone, Yoshiko Abe, Curtis W Marean, Peter J Nilssen,
Zelalem Assefa, and Elizabeth C Stone. 2010. ``A Review and Critique of
Quantification Procedures , and a New Image-Analysis GIS Approach.''
\emph{Amerian Antiquity} 67 (4): 643--63.

\hypertarget{ref-Aldeias2016}{}
Aldeias, Vera, Harold L. Dibble, Dennis Sandgathe, Paul Goldberg, and
Shannon J.P. McPherron. 2016. ``How heat alters underlying deposits and
implications for archaeological fire features: A controlled
experiment.'' \emph{Journal of Archaeological Science} 67. Elsevier Ltd:
64--79.
doi:\href{https://doi.org/10.1016/j.jas.2016.01.016}{10.1016/j.jas.2016.01.016}.

\hypertarget{ref-Aldeias2012}{}
Aldeias, Vera, Paul Goldberg, Dennis Sandgathe, Francesco Berna, Harold
L Dibble, Shannon P McPherron, Alain Turq, and Zeljko Rezek. 2012.
``Evidence for Neandertal use of fire at Roc de Marsal (France).''
\emph{Journal of Archaeological Science} 39 (7). Elsevier Ltd: 2414--23.
doi:\href{https://doi.org/10.1016/j.jas.2012.01.039}{10.1016/j.jas.2012.01.039}.

\hypertarget{ref-Alperson-Afil2008}{}
Alperson-Afil, Nira. 2008. ``Continual fire-making by Hominins at Gesher
Benot Ya'aqov, Israel.'' \emph{Quaternary Science Reviews} 27 (17-18):
1733--9.
doi:\href{https://doi.org/10.1016/j.quascirev.2008.06.009}{10.1016/j.quascirev.2008.06.009}.

\hypertarget{ref-Alperson-Afil2017a}{}
---------. 2017. ``Spatial Analysis of Fire: Archaeological Approach to
Recognizing Early Fire.'' \emph{Current Anthropology} 58 (S16):
S258--S266. doi:\href{https://doi.org/10.1086/692721}{10.1086/692721}.

\hypertarget{ref-Alperson-Afil2010a}{}
Alperson-Afil, Nira, and Naama Goren-Inbar. 2010. \emph{The Acheulian
Site of Gesher Benot Ya ` aqov}. Vol. II. New York.

\hypertarget{ref-Alperson-Afil2009}{}
Alperson-Afil, Nira, Gonen Sharon, Mordechai Kislev, Yoel Melamed, Irit
Zohar, Shosh Ashkenazi, Rivka Rabinovich, et al. 2009. ``Spatial
organization of hominin activities at gesher benot ya'aqov, Israel.''
\emph{Science} 326 (5960): 1677--80.
doi:\href{https://doi.org/10.1126/science.1180695}{10.1126/science.1180695}.

\hypertarget{ref-Anselin1995}{}
Anselin, Luc. 1995. ``Local indicators of spatial association ---
LISA.'' \emph{Geographical Analysis} 27 (2): 93--115.
doi:\href{https://doi.org/10.1111/j.1538-4632.1995.tb00338.x}{10.1111/j.1538-4632.1995.tb00338.x}.

\hypertarget{ref-Audouze1997}{}
Audouze, Françoise, and James G. Enloe. 1997. ``High resolution
archaeology at verberie: Limits and interpretations.'' \emph{World
Archaeology} 29 (2): 195--207.
doi:\href{https://doi.org/10.1080/00438243.1997.9980373}{10.1080/00438243.1997.9980373}.

\hypertarget{ref-Baales2001}{}
Baales, Michael. 2001. ``From lithics to spatial and social
organization: Interpreting the lithic distribution andraw material
composition at the final palaeolithic site of Kettig (Central Rhineland,
Geramny).'' \emph{Journal of Archaeological Science} 28 (2): 127--41.
doi:\href{https://doi.org/10.1006/jasc.1999.0545}{10.1006/jasc.1999.0545}.

\hypertarget{ref-Bailey2007}{}
Bailey, Geoff. 2007. ``Time perspectives, palimpsests and the
archaeology of time.'' \emph{Journal of Anthropological Archaeology} 26
(2): 198--223.
doi:\href{https://doi.org/10.1016/j.jaa.2006.08.002}{10.1016/j.jaa.2006.08.002}.

\hypertarget{ref-Bailey2009}{}
Bailey, Geoff, and Nena Galanidou. 2009. ``Caves palimpsests and
dwelling spaces: Examples from the Upper Palaeolithic of south-east
Europe.'' \emph{World Archaeology} 41 (2): 215--41.
doi:\href{https://doi.org/10.1080/00438240902843733}{10.1080/00438240902843733}.

\hypertarget{ref-Bamforth2005}{}
Bamforth, Douglas B, Mark Becker, and Jean Hudson. 2005. ``Intrasite
Spatial Analysis, Ethnoarchaeology , and Paleoindian Land-Use on the
Great Plains: The Allen Site.'' \emph{Amerian Antiquity} 70 (3):
561--80.

\hypertarget{ref-Bargallo2016}{}
Bargalló, Amèlia, Maria Joana Gabucio, and Florent Rivals. 2016.
``Puzzling out a palimpsest: Testing an interdisciplinary study in level
O~of Abric Romaní.'' \emph{Quaternary International} 417: 51--65.
doi:\href{https://doi.org/10.1016/j.quaint.2015.09.066}{10.1016/j.quaint.2015.09.066}.

\hypertarget{ref-Baxter1997}{}
Baxter, M. J., C. C. Beardah, and R. V.S. Wright. 1997. ``Some
archaeological applications of kernel density estimates.'' \emph{Journal
of Archaeological Science} 24 (4): 347--54.
doi:\href{https://doi.org/10.1006/jasc.1996.0119}{10.1006/jasc.1996.0119}.

\hypertarget{ref-Benito-calvo2011}{}
Benito-Calvo, Alfonso, and Ignacio de la Torre. 2011. ``Analysis of
orientation patterns in Olduvai Bed I assemblages using GIS techniques :
Implications for site formation processes.'' \emph{Journal of Human
Evolution} 61 (1). Elsevier Ltd: 50--60.
doi:\href{https://doi.org/10.1016/j.jhevol.2011.02.011}{10.1016/j.jhevol.2011.02.011}.

\hypertarget{ref-Binford1978}{}
Binford, Lewis R. 1978. ``Dimensional Analysis of Behavior and Site
Structure : Learning from an Eskimo Hunting Stand Author (.''
\emph{American Antiquity} 43 (3): 330--61.

\hypertarget{ref-Binford1981a}{}
---------. 1981. ``Behavioral Archaeology and the `Pompei Premise'.''
\emph{Journal of Anthropological Research} 37 (3): 195--208.
doi:\href{https://doi.org/10.1017/CBO9781107415324.004}{10.1017/CBO9781107415324.004}.

\hypertarget{ref-Bisson2014}{}
Bisson, Michael S., April Nowell, Carlos Cordova, Melanie Poupart, and
Christopher Ames. 2014. ``Dissecting palimpsests in a Late Lower and
Middle Paleolithic flint acquisition site on the Madaba Plateau,
Jordan.'' \emph{Quaternary International} 331. Elsevier Ltd; INQUA:
74--94.
doi:\href{https://doi.org/10.1016/j.quaint.2013.05.031}{10.1016/j.quaint.2013.05.031}.

\hypertarget{ref-Bivand2013}{}
Bivand, Roger, Edzer Pebesma, and Virgillio Gómez-Rubio. 2013.
\emph{Applied Spatial Data Analysis with R}. New York: Springer.
doi:\href{https://doi.org/10.1007/978-0-387-78171-6}{10.1007/978-0-387-78171-6}.

\hypertarget{ref-Blasco2016}{}
Blasco, Ruth, Jordi Rosell, Pablo Sañudo, Avi Gopher, and Ran Barkai.
2016. ``What happens around a fire: Faunal processing sequences and
spatial distribution at Qesem Cave (300 ka), Israel.'' \emph{Quaternary
International} 398: 190--209.
doi:\href{https://doi.org/10.1016/j.quaint.2015.04.031}{10.1016/j.quaint.2015.04.031}.

\hypertarget{ref-bordes_decouverte_1962}{}
Bordes, F, and J Lafille. 1962. ``Découverte d'un Squelette d'enfant
Moustérien Dans Le Gisement de Roc de Marsal, Commune de
Campagne-Du-Bugue (Dordogne).'' \emph{CR Acad Sci Paris} 254: 714--15.

\hypertarget{ref-Brooks1987}{}
Brooks, Alison, and John Yellen. 1987. ``The Preservation of Activity
Areas in the Archaeological Record: Ethnoarchaeological and
Archaeological Work in Northwest Ngamiland, Botswana.'' In \emph{Method
and Theory for Activity Area Research: An Ethnoarchaeological Approach},
63--106. New York: Columbia University Press.

\hypertarget{ref-Carr1984}{}
Carr, Chris. 1984. ``The Nature of Organization of Intrasite
Archaeological Records and Spatial Analytic Approaches to Their
Investigation.'' In \emph{Advances in Archaeological Method and Theory
7}, 103--22. Orlando: Academic Press.

\hypertarget{ref-castel_neandertal_2017}{}
Castel, Jean-Christophe, Emmanuel Discamps, Marie-Cécile Soulier, Dennis
Sandgathe, Harold L Dibble, Shannon J P McPherron, Paul Goldberg, and
Alain Turq. 2017. ``Neandertal Subsistence Strategies during the Quina
Mousterian at Roc de Marsal (France).'' \emph{Cleaning up a Messy
Mousterian: How to Describe and Interpret Late Middle Palaeolithic
Chrono-Cultural Variability in Atlantic Europe} 433 (March): 140--56.
doi:\href{https://doi.org/10.1016/j.quaint.2015.12.033}{10.1016/j.quaint.2015.12.033}.

\hypertarget{ref-Clark2016}{}
Clark, Amy E. 2016. ``Time and space in the middle paleolithic: Spatial
structure and occupation dynamics of seven open-air sites.''
\emph{Evolutionary Anthropology} 25 (3): 153--63.
doi:\href{https://doi.org/10.1002/evan.21486}{10.1002/evan.21486}.

\hypertarget{ref-Clark2017}{}
---------. 2017. ``From Activity Areas to Occupational Histories: New
Methods to Document the Formation of Spatial Structure in
Hunter-Gatherer Sites.'' \emph{Journal of Archaeological Method and
Theory} 24 (4). Journal of Archaeological Method; Theory: 1300--1325.
doi:\href{https://doi.org/10.1007/s10816-017-9313-7}{10.1007/s10816-017-9313-7}.

\hypertarget{ref-clarkson_archaeology_2015}{}
Clarkson, Chris, Mike Smith, Ben Marwick, Richard Fullagar, Lynley A
Wallis, Patrick Faulkner, Tiina Manne, et al. 2015. ``The Archaeology,
Chronology and Stratigraphy of Madjedbebe (Malakunanja II): A Site in
Northern Australia with Early Occupation.'' \emph{Journal of Human
Evolution} 83 (June): 46--64.
doi:\href{https://doi.org/10.1016/j.jhevol.2015.03.014}{10.1016/j.jhevol.2015.03.014}.

\hypertarget{ref-dibble_measurement_2005}{}
Dibble, Harold L, Utsav A Schurmans, Radu P Iovita, and Michael V
McLaughlin. 2005. ``The Measurement and Interpretation of Cortex in
Lithic Assemblages.'' \emph{American Antiquity} 70 (3): 545--60.
doi:\href{https://doi.org/10.2307/40035313}{10.2307/40035313}.

\hypertarget{ref-Dibble1997}{}
Dibble, Harold L., Philip G. Chase, Shannon P. McPherron, and Alain
Tuffreau. 1997. ``Testing the Reality of a ` Living Floor ' with
Archaeological Data.'' \emph{American Antiquity} 62 (4): 629--51.

\hypertarget{ref-Douglass2015}{}
Douglass, Matthew J., Simon J. Holdaway, Patricia C. Fanning, and Justin
I. Shiner. 2015. ``An Assessment and Archaeological Application of
Cortex Measurment in Lithic Assemblages.'' \emph{Amerian Antiquity} 19
(1): 64--82.

\hypertarget{ref-Ebert1992}{}
Ebert, James I. 1992. \emph{Distriubtional Archaeology}. Salt Lake City:
University of Utah Press.

\hypertarget{ref-Foley1981a}{}
Foley, Robert. 1981. ``Off-site archaeology: an alternative approach for
the short-sited.'' In \emph{Pattern of the Past: Studies in Honour of
David Clarke}, 33:139--53. Cambridge: Cambridge University Press.

\hypertarget{ref-Galanidou1997}{}
Galanidou, Nena. 1997. \emph{Home is weather the hearth is: The Spatial
Organisation of the Upper Palaeolithic Rockshelter Occupation at Klithi
and Kastritsa in Northwest Greece.} Oxford: BAR Internationla series.

\hypertarget{ref-goldberg_testing_2013}{}
Goldberg, Paul, Vera Aldeias, Harold Dibble, Shannon McPherron, Dennis
Sandgathe, and Alain Turq. 2013. ``Testing the Roc de Marsal Neandertal
"Burial'' with Geoarchaeology.'' \emph{Archaeological and
Anthropological Sciences}, 1--11.
doi:\href{https://doi.org/10.1007/s12520-013-0163-2}{10.1007/s12520-013-0163-2}.

\hypertarget{ref-Goldberg2010}{}
Goldberg, Paul, and Francesco Berna. 2010. ``Micromorphology and
context.'' \emph{Quaternary International} 214 (1-2). Elsevier Ltd;
INQUA: 56--62.
doi:\href{https://doi.org/10.1016/j.quaint.2009.10.023}{10.1016/j.quaint.2009.10.023}.

\hypertarget{ref-goldberg_new_2012}{}
Goldberg, Paul, Harold Dibble, Francesco Berna, Dennis Sandgathe,
Shannon J P McPherron, and Alain Turq. 2012. ``New Evidence on
Neandertal Use of Fire: Examples from Roc de Marsal and Pech de l'Azé
IV.'' \emph{The Neanderthal Home: Spatial and Social Behaviours} 247
(0): 325--40.
doi:\href{https://doi.org/10.1016/j.quaint.2010.11.015}{10.1016/j.quaint.2010.11.015}.

\hypertarget{ref-Gopher2016}{}
Gopher, Avi, Yoni Parush, Ella Assaf, and Ran Barkai. 2016. ``Spatial
aspects as seen from a density analysis of lithics at Middle Pleistocene
Qesem Cave: Preliminary results and observations.'' \emph{Quaternary
International} 398. Elsevier Ltd: 103--17.
doi:\href{https://doi.org/10.1016/j.quaint.2015.09.078}{10.1016/j.quaint.2015.09.078}.

\hypertarget{ref-Gould1987}{}
Gould, Richard A., and John E. Yellen. 1987. ``Man the hunted:
Determinants of household spacing in desert and tropical foraging
societies.'' \emph{Journal of Anthropological Archaeology} 6 (1):
77--103.
doi:\href{https://doi.org/10.1016/0278-4165(87)90017-1}{10.1016/0278-4165(87)90017-1}.

\hypertarget{ref-guerin_multi-method_2012}{}
Guérin, Guillaume, Emmanuel Discamps, Christelle Lahaye, Norbert
Mercier, Pierre Guibert, Alain Turq, Harold L Dibble, et al. 2012.
``Multi-Method (TL and OSL), Multi-Material (Quartz and Flint) Dating of
the Mousterian Site of Roc de Marsal (Dordogne, France): Correlating
Neanderthal Occupations with the Climatic Variability of MIS-3.''
\emph{Journal of Archaeological Science} 39 (10): 3071--84.
doi:\href{https://doi.org/10.1016/j.jas.2012.04.047}{10.1016/j.jas.2012.04.047}.

\hypertarget{ref-guerin_complementarity_2017}{}
Guérin, Guillaume, Marine Frouin, Joséphine Tuquoi, Kristina J Thomsen,
Paul Goldberg, Vera Aldeias, Christelle Lahaye, et al. 2017. ``The
Complementarity of Luminescence Dating Methods Illustrated on the
Mousterian Sequence of the Roc de Marsal: A Series of
Reindeer-Dominated, Quina Mousterian Layers Dated to MIS 3.''
\emph{Cleaning up a Messy Mousterian: How to Describe and Interpret Late
Middle Palaeolithic Chrono-Cultural Variability in Atlantic Europe} 433
(March): 102--15.
doi:\href{https://doi.org/10.1016/j.quaint.2016.02.063}{10.1016/j.quaint.2016.02.063}.

\hypertarget{ref-guibert_importance_2009}{}
Guibert, Pierre, Christelle Lahaye, and Françoise Bechtel. 2009. ``The
Importance of U-Series Disequilibrium of Sediments in Luminescence
Dating: A Case Study at the Roc de Marsal Cave (Dordogne, France).''
\emph{Radiation Measurements} 44: 223--31.

\hypertarget{ref-Hagen-Zanker2016}{}
Hagen-Zanker, Alex. 2016. ``A computational framework for generalized
moving windows and it sapplication to landscape pattern analysis.''
\emph{International Journal of Applied Earth Observation and
Geoinformation} 44. Elsevier B.V.: 205--16.
doi:\href{https://doi.org/10.1016/j.jag.2015.09.010}{10.1016/j.jag.2015.09.010}.

\hypertarget{ref-Henry2012}{}
Henry, Donald. 2012. ``The palimpsest problem, hearth pattern analysis,
and Middle Paleolithic site structure.'' \emph{Quaternary International}
247 (1). Elsevier Ltd; INQUA: 246--66.
doi:\href{https://doi.org/10.1016/j.quaint.2010.10.013}{10.1016/j.quaint.2010.10.013}.

\hypertarget{ref-Henry2004}{}
Henry, Donald O, Harold J Hietala, Arlene M Rosen, Yuri E Demidenko,
Vitaliy I Usik, and Teresa L Armagan. 2004. ``Human Behavioral
Organization in the Middle Paleolithic : Were Neanderthals Different ?''
106 (1): 17--31.

\hypertarget{ref-hodgkins_climate-mediated_2016}{}
Hodgkins, Jamie, Curtis W Marean, Alain Turq, Dennis Sandgathe, Shannon
J P McPherron, and Harold Dibble. 2016. ``Climate-Mediated Shifts in
Neandertal Subsistence Behaviors at Pech de l'Azé IV and Roc de Marsal
(Dordogne Valley, France).'' \emph{Journal of Human Evolution} 96
(July): 1--18.
doi:\href{https://doi.org/10.1016/j.jhevol.2016.03.009}{10.1016/j.jhevol.2016.03.009}.

\hypertarget{ref-Holdaway2008}{}
Holdaway, Simon, and Luann Wandsnider. 2008. \emph{Time in Archaeology:
Time Perspectivism Revisited}. Salt Lake City: University of Utah Press.

\hypertarget{ref-Isserman1977a}{}
Isserman, Andrew M. 1977. ``The Location Quotient Approach to Estimating
Regional Economic Impacts.'' \emph{Journal of the American Planning
Association} 43 (1): 33--41.
doi:\href{https://doi.org/10.1080/01944367708977758}{10.1080/01944367708977758}.

\hypertarget{ref-Kroll1991}{}
Kroll, Ellen M., and T. Douglas Price. 1991. \emph{The Interpretation of
Archaeological Spatial Patterning}. New York: Springer.

\hypertarget{ref-Leroi-Gourhan1984}{}
Leroi-Gourhan, A. 1984. ``Pincevent: campement magdalenien de chasseurs
de rennes, Guides archaeologiques de la France.'' \emph{Ministere de La
Culture Direction Du Patrimoine Sous-Directon de L'archeologie, Paris}.

\hypertarget{ref-lin_establishing_2015}{}
Lin, Sam C, Shannon P McPherron, and Harold L Dibble. 2015.
``Establishing Statistical Confidence in Cortex Ratios within and among
Lithic Assemblages: A Case Study of the Middle Paleolithic of
Southwestern France.'' \emph{Journal of Archaeological Science} 59
(July): 89--109.
doi:\href{https://doi.org/10.1016/j.jas.2015.04.004}{10.1016/j.jas.2015.04.004}.

\hypertarget{ref-Lin2016}{}
Lin, Sam C., Cornel M. Pop, Harold L. Dibble, Will Archer, Dawit Desta,
Marcel Weiss, and Shannon P. McPherron. 2016. ``A Core Reduction
Experiment Finds No Effect of Original Stone Size and Reduction
Intensity on Flake Debris Size Distribution.'' \emph{American Antiquity}
81 (03): 562--75.
doi:\href{https://doi.org/10.1017/S0002731600004005}{10.1017/S0002731600004005}.

\hypertarget{ref-Lloyd2006}{}
Lloyd, C D. 2006. \emph{Local Models for Spatial Analysis}. London: CRC
Press.

\hypertarget{ref-Luncz2016}{}
Luncz, Lydia V., Tomos Proffitt, Lars Kulik, Michael Haslam, and Roman
M. Wittig. 2016. ``Distance-decay effect in stone tool transport by wild
chimpanzees.'' \emph{Proceedings of the Royal Society B: Biological
Sciences} 283 (1845): 20161607.
doi:\href{https://doi.org/10.1098/rspb.2016.1607}{10.1098/rspb.2016.1607}.

\hypertarget{ref-Machado2016}{}
Machado, Jorge, Francisco J. Molina, Cristo M. Hernández, Antonio
Tarriño, and Bertila Galván. 2016. ``Using lithic assemblage formation
to approach Middle Palaeolithic settlement dynamics: El Salt
Stratigraphic Unit X (Alicante, Spain).'' \emph{Archaeological and
Anthropological Sciences}, 1--29.
doi:\href{https://doi.org/10.1007/s12520-016-0318-z}{10.1007/s12520-016-0318-z}.

\hypertarget{ref-Malinsky-Buller2011}{}
Malinsky-Buller, Ariel, Erella Hovers, and Ofer Marder. 2011. ``Making
time: 'Living floors', 'palimpsests' and site formation processes - A
perspective from the open-air Lower Paleolithic site of Revadim Quarry,
Israel.'' \emph{Journal of Anthropological Archaeology} 30 (2). Elsevier
Inc.: 89--101.
doi:\href{https://doi.org/10.1016/j.jaa.2010.11.002}{10.1016/j.jaa.2010.11.002}.

\hypertarget{ref-Mallol2016}{}
Mallol, Carolina, and Cristo Hernández. 2016. ``Advances in palimpsest
dissection.'' \emph{Quaternary International} 417: 1--2.
doi:\href{https://doi.org/10.1016/j.quaint.2016.09.021}{10.1016/j.quaint.2016.09.021}.

\hypertarget{ref-Marwick2017}{}
Marwick, Ben. 2017. ``Computational Reproducibility in Archaeological
Research: Basic Principles and a Case Study of Their Implementation.''
\emph{Journal of Archaeological Method and Theory} 24 (2). Journal of
Archaeological Method; Theory: 424--50.
doi:\href{https://doi.org/10.1007/s10816-015-9272-9}{10.1007/s10816-015-9272-9}.

\hypertarget{ref-McCoy2001}{}
McCoy, Jill, and Kevin Johnston. 2001. \emph{Using ArcGIS spatial
analyst: GIS by ESRI}.

\hypertarget{ref-McPherron2005a}{}
McPherron, Shannon J.P., Harold L. Dibble, and Paul Goldberg. 2005.
``Z.'' \emph{Geoarchaeology} 20 (3): 243--62.
doi:\href{https://doi.org/10.1002/gea.20048}{10.1002/gea.20048}.

\hypertarget{ref-mcpherron_additional_2018}{}
McPherron, Shannon P. 2018. ``Additional Statistical and Graphical
Methods for Analyzing Site Formation Processes Using Artifact
Orientations.'' \emph{PLOS ONE} 13 (1): e0190195.
doi:\href{https://doi.org/10.1371/journal.pone.0190195}{10.1371/journal.pone.0190195}.

\hypertarget{ref-Mentzer2009}{}
Mentzer, Susan M. 2009. ``Bone as a fuel source: the effects of initial
fragment size distribution.'' In \emph{Gestion Des Combustibles Au
Paleolithique et Au Mesolithique: Nouveaux Outiles, Nouvelles
Interpretations. Uispp Proceedings of the Xv World Congress (Lisbon,
4--9 September 2006)}. Oxford: Archaeopress.

\hypertarget{ref-Merrill2010}{}
Merrill, Michael, and Dwitght Read. 2010. ``A New Method Using Graph and
Lattice Theory to Discover Spatial Cohesive Sets of Artifacts and Areas
of Organized Activity in Archaeological Sites.'' \emph{American
Antiquity} 75 (3): 419--51.

\hypertarget{ref-Miller2007}{}
Miller, John H, and Scott E Page. 2007. \emph{Complex Adaptive Systems:
An Introduction to Computational Models of Social Life}. Vol. 27.
Princeton: Princeton University Press.
doi:\href{https://doi.org/10.1016/S1460-1567(08)10011-3}{10.1016/S1460-1567(08)10011-3}.

\hypertarget{ref-Nielsen1991}{}
Nielsen, Axel E. 1991. ``Trampling the Archaeological Record: An
Experimental Study.'' \emph{American Antiquity} 56 (3): 483.
doi:\href{https://doi.org/10.2307/280897}{10.2307/280897}.

\hypertarget{ref-Oron2014}{}
Oron, Maya, and Naama Goren-Inbar. 2014. ``Mousterian intra-site spatial
patterning at quneitra, golan heights.'' \emph{Quaternary International}
331. Elsevier Ltd; INQUA: 186--202.
doi:\href{https://doi.org/10.1016/j.quaint.2013.04.013}{10.1016/j.quaint.2013.04.013}.

\hypertarget{ref-OConnell1991}{}
O'Connell, James, Kristen Hawkes, and Nicholas Blurton Jones. 1991.
``Distribution of Refuse-Producing Activities at Hadza Residential Base
Camps: Implications for Analyses of Archaeological Site Structure.'' In
\emph{The Interpretation of Archaeological Site Patterning}, 61--75. New
York: Springer.

\hypertarget{ref-Pettitt1997a}{}
Pettitt, P. B. 1997. ``High resolution neanderthals? interpreting middle
palaeolithic intrasite spatial data.'' \emph{World Archaeology} 29 (2):
208--24.
doi:\href{https://doi.org/10.1080/00438243.1997.9980374}{10.1080/00438243.1997.9980374}.

\hypertarget{ref-Reed2015}{}
Reed, Denné, W. Andrew Barr, Shannon P. Mcpherron, René Bobe, Denis
Geraads, Jonathan G. Wynn, and Zeresenay Alemseged. 2015. ``Digital data
collection in paleoanthropology.'' \emph{Evolutionary Anthropology} 24
(6): 238--49.
doi:\href{https://doi.org/10.1002/evan.21466}{10.1002/evan.21466}.

\hypertarget{ref-Reynard2018}{}
Reynard, Jerome P, and Christopher S Henshilwood. 2018. ``Using
Trampling Modification to Infer Occupational Intensity During the Still
Bay at Blombos Cave , Southern Cape , South Africa.'' African
Archaeological Review.

\hypertarget{ref-Riel-Salvatore2013}{}
Riel-Salvatore, Julien, Ingrid. C Ludeke, Fabio Negrino, and Brigitte.M
Holt. 2013. ``A Spatial Analysis of the Late Mousterian Levels of Riparo
Bombrini (Balzi Rossi, Italy).'' \emph{Canadian Journal of Archaeology}
92: 70--92.

\hypertarget{ref-RodaGilabert2016}{}
Roda Gilabert, Xavier, Jorge Martínez-Moreno, and Rafael Mora Torcal.
2016. ``Ground stone tools and spatial organization at the Mesolithic
site of font del Ros (southeastern Pre-Pyrenees, Spain).'' \emph{Journal
of Archaeological Science: Reports} 5. Elsevier Ltd: 209--24.
doi:\href{https://doi.org/10.1016/j.jasrep.2015.11.023}{10.1016/j.jasrep.2015.11.023}.

\hypertarget{ref-sandgathe_roc_2011}{}
Sandgathe, Dennis M, Harold L Dibble, Paul Goldberg, and Shannon P
McPherron. 2011. ``The Roc de Marsal Neandertal Child: A Reassessment of
Its Status as a Deliberate Burial.'' \emph{Journal of Human Evolution}
61 (3): 243--53.
doi:\href{https://doi.org/10.1016/j.jhevol.2011.04.003}{10.1016/j.jhevol.2011.04.003}.

\hypertarget{ref-sandgathe_role_2011}{}
Sandgathe, Dennis M, Harold L Dibble, Paul Goldberg, Shannon P
McPherron, Alain Turq, Laura Niven, and Jamie Hodgkins. 2011. ``On the
Role of Fire in Neandertal Adaptations in Western Europe: Evidence from
Pech de l'Azé and Roc de Marsal, France.'' \emph{PaleoAnthropology}
2011: 216--42.

\hypertarget{ref-sandgathe_introduction_2018}{}
Sandgathe, Dennis M, Harold L Dibble, Shannon J P McPherron, and Paul
Goldberg. 2018. ``Introduction.'' In \emph{The Middle Paleolithic Site
of Pech de L'Azé Iv}, 1--19. Cave and Karst Systems of the World.
Springer, Cham.
doi:\href{https://doi.org/10.1007/978-3-319-57524-7_1}{10.1007/978-3-319-57524-7\_1}.

\hypertarget{ref-Sandrine2005}{}
Sandrine, Costamagno, Théry-Parisot Isabelle, Jean Philip Brugal, and
Raphaele Guibert. 2005. ``Taphonomic consequences of the use of bones as
fuel. Experimental data and archaeological applications.'' In
\emph{Biosphere to Lithosphere, Proceedings of the 9th Conference of the
International Council of Archaeozoology}, 51--62. Oxford: Oxbow books.

\hypertarget{ref-Schelling1978}{}
Schelling, Thomas C. 1978. \emph{Micromotives and Macrobehaviors}.
Toronto: W. W. Norton \& Company.
doi:\href{https://doi.org/10.2307/2989930}{10.2307/2989930}.

\hypertarget{ref-Schiffer1975}{}
Schiffer, Michael B. 1975. ``Archaeology as Behavioral Science.''
\emph{American Anthropologist} 77 (4): 836--48.
doi:\href{https://doi.org/10.1525/aa.1975.77.4.02a00060}{10.1525/aa.1975.77.4.02a00060}.

\hypertarget{ref-Shott1998}{}
Shott, Michael J. 1998. ``Lower Paleolithic Industries , Time , and the
Meaning of Assemblage Variation.'' \emph{Time in Archaeology}, 46--60.

\hypertarget{ref-Simms1990}{}
Simms, Steven R, and Kathleen M Heath. 1990. ``Site Structure of the
Orbit Inn : An Application of Ethnoarchaeology.'' \emph{Amerian
Antiquity} 55 (4): 797--813.

\hypertarget{ref-Stern1994}{}
Stern, Nicola. 1994. ``The implications of time-averaging for
reconstructing the land-use patterns of early tool-using hominids.''
\emph{Journal of Human Evolution} 27 (1-3): 89--105.
doi:\href{https://doi.org/10.1006/jhev.1994.1037}{10.1006/jhev.1994.1037}.

\hypertarget{ref-Stern1993}{}
Stern, Nicola, Henry T. Bunn, Ellen M. Kroll, Gary Haynes, Sally
McBrearty, Jeanne Sept, and Pamela R. Willoughby. 1993. ``The Structure
of the Lower Pleistocene Archaeological Record: A Case Study From the
Koobi Fora Formation {[}and Comments and Reply{]}.'' \emph{Current
Anthropology} 34 (3): 201--25.
doi:\href{https://doi.org/10.1086/204164}{10.1086/204164}.

\hypertarget{ref-Toth1985}{}
Toth, Nicholas. 1985. ``The oldowan reassessed: A close look at early
stone artifacts.'' \emph{Journal of Archaeological Science} 12 (2):
101--20.
doi:\href{https://doi.org/10.1016/0305-4403(85)90056-1}{10.1016/0305-4403(85)90056-1}.

\hypertarget{ref-jaubert_mousterien_2008}{}
Turq, A, H Dibble, J.-P. Faivre, P Goldberg, S J P McPherron, and D
Sandgathe. 2008. ``Le Moustérien Récent Du Périgord Noir: Quoi De Neuf
?'' In \emph{Les Sociétés Du Paléolithique Dans Un Grand Sud-Ouest de La
France: Nouveaux Gisements, Nouveaux Résultats, Nouvelles Méthodes},
edited by J Jaubert, J.-G. Bordes, and I Ortega, 83--94. Mémoire de la
Société Préhistorique Française 48.

\hypertarget{ref-turq_evolution_1979}{}
Turq, Alain. 1979. ``L'evolution Du Mousterian de Type Quina Au Roc de
Marsal et En Perigord. Modifications de l'équilibre Technique et
Typologique.'' Mémoire., L'Ecole des Hautes Etudes en Sciences Sociales.

\hypertarget{ref-turq_mousterien_1985}{}
---------. 1985. ``Le Moustérien de Type Quina Du Roc de Marsal
(Dordogne).'' \emph{Bulletin de La Société Préhistorique Française} 82
(2): 46--51.

\hypertarget{ref-Vaquero2012}{}
Vaquero, Manuel, Susana Alonso, Sergio García-Catalán, Angélica
García-Hernández, Bruno Gómez de Soler, David Rettig, and María Soto.
2012. ``Temporal nature and recycling of Upper Paleolithic artifacts:
The burned tools from the Molí del Salt site (Vimbodí i Poblet,
northeastern Spain).'' \emph{Journal of Archaeological Science} 39 (8).
Elsevier Ltd: 2785--96.
doi:\href{https://doi.org/10.1016/j.jas.2012.04.024}{10.1016/j.jas.2012.04.024}.

\hypertarget{ref-Vaquero2001}{}
Vaquero, Manuel, and Ignasi Pastó. 2001. ``The Definition of Spatial
Units in Middle Palaeolithic Sites: The Hearth-Related Assemblages.''
\emph{Journal of Archaeological Science} 28 (11): 1209--20.
doi:\href{https://doi.org/10.1006/jasc.2001.0656}{10.1006/jasc.2001.0656}.

\hypertarget{ref-Vaquero2012a}{}
Vaquero, Manuel, María Gema Chacón, Maria Dolores García-Antón, Bruno
Gómez de Soler, Kenneth Martínez, and Felipe Cuartero. 2012. ``Time and
space in the formation of lithic assemblages: The example of Abric
Romaní Level J.'' \emph{Quaternary International} 247 (1). Elsevier Ltd;
INQUA: 162--81.
doi:\href{https://doi.org/10.1016/j.quaint.2010.12.015}{10.1016/j.quaint.2010.12.015}.

\hypertarget{ref-Wandsnider2008}{}
Wandsnider, LuAnn. 2008. ``Time-Averaged Deposits and Multitemporal
Processes in the Wyoming Basin, Intermontane North America: A
Preliminary Consideration of Land Tenure in Terms of Occupation
Frequency and Integration.'' \emph{Time in Archaeology: Time
Perspectivism Revisited}, 61--93.

\hypertarget{ref-Wheatley}{}
Wheatley, David, and Mark Gillings. 2013. \emph{Spatial technology and
archaeology: the archaeological applications of GIS.} New York: CRC
Press.

\hypertarget{ref-Yellen1977}{}
Yellen, John E. 1977. \emph{Archaeological approaches to the present :
models for reconstructing the past}. New York: Academic Press.

\hypertarget{ref-Yvorra2003}{}
Yvorra, Pascale. 2003. ``The management of space in a Palaeolithic rock
shelter: Defining activity areas by spatial analysis.'' \emph{Antiquity}
77 (296): 336--44.
doi:\href{https://doi.org/10.1017/S0003598X00092310}{10.1017/S0003598X00092310}.

\bibliographystyle{spbasic}
\bibliography{RDM\_BIB.bib}

\end{document}
